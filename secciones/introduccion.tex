\intro{

  La explosión de la información, así como la aparición de una gran variedad de
  formatos en los cuales se presenta, son dos características importantes de la
  llamada isociedad de la información, es en este contexto en el cual resulta
  necesario que las personas, las cuales requieren de información a diario,
  desarrollen habilidades de acceso, evaluación, uso y comunicación de la
  información de tal modo que sean capaces de aprovecharla y utilizarla
  eficazmente en diversos aspectos de su vidas.

  Ante esta situación, las bibliotecas tienen un rol que cumplir pues el
  carácter educativo que las identifica puede ser beneficioso si los
  bibliotecarios trabajan de manera participativa en la formación de los
  estudiantes colaborando así con el éxito académico. El rol que debe asumir el
  bibliotecario se centra en lograr que los estudiantes posean competencias
  informativas teniendo presente que un alumno que posee habilidades de
  búsqueda, análisis y uso de la información estará en ventaja sobre otros, lo
  cual le permitirá, sin duda, desarrollarse con éxito.

  Ante esta situación, las bibliotecas tienen un rol que cumplir pues el
  carácter educativo que las identifica puede ser beneficioso si los
  bibliotecarios trabajan de manera participativa en la formación de los
  estudiantes colaborando así con el éxito académico. El rol que debe asumir el
  bibliotecario se centra en lograr que los estudiantes posean competencias
  informativas teniendo presente que un alumno que posee habilidades de
  búsqueda, análisis y uso de la información estará en ventaja sobre otros, lo
  cual le permitirá, sin duda, desarrollarse con éxito.

  Ante esta situación, las bibliotecas tienen un rol que cumplir pues el
  carácter educativo que las identifica puede ser beneficioso si los
  bibliotecarios trabajan de manera participativa en la formación de los
  estudiantes colaborando así con el éxito académico. El rol que debe asumir el
  bibliotecario se centra en lograr que los estudiantes posean competencias
  informativas teniendo presente que un alumno que posee habilidades de
  búsqueda, análisis y uso de la información estará en ventaja sobre otros, lo
  cual le permitirá, sin duda, desarrollarse con éxito.

  Continúe aquí el texto xxxxx

}
